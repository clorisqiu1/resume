% !TEX TS-program = xelatex
% !TEX encoding = UTF-8 Unicode
% !Mode:: "TeX:UTF-8"

\documentclass{resume}
\usepackage{zh_CN-Adobefonts_external} % Simplified Chinese Support using external fonts (./fonts/zh_CN-Adobe/)
% \usepackage{NotoSansSC_external}
% \usepackage{NotoSerifCJKsc_external}
% \usepackage{zh_CN-Adobefonts_internal} % Simplified Chinese Support using system fonts
\usepackage{linespacing_fix} % disable extra space before next section
\usepackage{cite}

\begin{document}
\pagenumbering{gobble} % suppress displaying page number

\name{邱雯}

\basicInfo{
  \email{clorisqiu1@gmail.com} \textperiodcentered\ 
  \phone{(+86) 178-5559-3386} \textperiodcentered\ 
  \github[MyPage]{https://clorisqiu1.github.io/}}
 
\section{\faGraduationCap\  教育背景}
\datedsubsection{\textbf{北見工業大学}, 北海道, 日本}{2020 -- 至今}
\textit{博士在读}\ 生产基盘工学专攻 (Doctoral Course in Manufacturing Engineering), 预计 2024 年 10 月毕业
\datedsubsection{\textbf{安徽工业大学}, 安徽}{2017.09 -- 2019.12}
\textit{硕士}\ 管理科学与工程

\section{\faInfo\ 研究经历}
\datedsubsection{\textbf{北見工業大学}, 北海道, 日本}{2020 -- 至今}
在 核科学电子情报工学实验室 (Information Technology for Nuclear Science Laboratory) 从事博士研究生研究工作, 主要研究方向为强化学习及其在无线通信和网络优化中的应用.
\datedsubsection{\textbf{安徽工业大学}, 安徽}{2017.09 -- 2019.12}
在赵伟副教授指导下从事无线通信和网络优化领域的研究工作, 主要研究方向为边缘计算及其在无线通信和网络优化中的应用.

% \end{onehalfspacing}

% \datedsubsection{\textbf{\LaTeX\ 简历模板}}{2015 年5月 -- 至今}
% \role{\LaTeX, Python}{个人项目}
% \begin{onehalfspacing}
% 优雅的 \LaTeX\ 简历模板, https://github.com/billryan/resume
% \begin{itemize}
%   \item 容易定制和扩展
%   \item 完善的 Unicode 字体支持,使用 \XeLaTeX\ 编译
%   \item 支持 FontAwesome 4.5.0
% \end{itemize}
% \end{onehalfspacing}

% Reference Test
%\datedsubsection{\textbf{Paper Title\cite{zaharia2012resilient}}}{May. 2015}
%An xxx optimized for xxx\cite{verma2015large}
%\begin{itemize}
%  \item main contribution
%\end{itemize}

\section{\faCogs\ 论文发表情况}
% increase linespacing [parsep=0.5ex]
\begin{itemize}[parsep=0.5ex]
  \item Edge-node assisted live video streaming: A coalition formation game approach, W Zhao, \textbf{W Qiu}, C Zhou, Z Liu, T Hara, 2018 IEEE Globecom Workshops (GC Wkshps), 2018. (二作)
  \item Understanding World Models through Multi-Step Pruning Policy via Reinforcement Learning, ZhiQiang He, \textbf{Wen Qiu} et.al., IEEE Transactions on Emerging Topics in Computing (2024). (通讯作者, 二作, 在投)
  \item Optimizing UAV-Assisted Emergency Communications with MAPPO: A Dynamic and Fair Approach, \textbf{Wen Qiu}, Hiroshi Masui et.al, Journal of Circuits, Systems and Computers (2024). (一作, 在投)
  \item Optimization of UAV-Assisted Emergency Scenario Communication Considering Energy Consumption, \textbf{Wen Qiu}, Hiroshi Masui et.al, IEEE ISPA 2024. (一作, 在投)
\end{itemize}
% \section{\faHeartO\ 获奖情况}
% \datedline{\textit{第一名}, xxx 比赛}{2013 年6 月}
% \datedline{其他奖项}{2015}

\section{\faHeartO\ 其他获奖情况}
% increase linespacing [parsep=0.5ex]
% \begin{itemize}[parsep=0.5ex]
  % \item 技术博客: $https://blog.csdn.net/weixin_41794514?type=blog$
  % \item GitHub: https://github.com/username
  % \item 语言: 英语 - 熟练, 日语 - 初级
% \end{itemize}
\datedsubsection{\textbf{安徽工业大学}, 安徽}{2017年9月 -- 2019年12月}
\begin{itemize}
  \item 2018 “华为杯”第 15 届中国研究生数学建模竞赛全国三等奖
  \item 2018 全国大学生信息安全竞赛安徽省赛大数据与人工智能应用类一等奖
  \item 2019 国家奖学金
  \item 2017 第二届全国高等院校项目管理大赛(研究生组)团体赛全国一等奖
  \item 2017 第二届全国高等院校项目管理大赛(研究生组)个人赛全国三等奖
  \item 2018 第三届全国高等院校项目管理大赛(研究生组)个人赛全国三等奖
  \item 2017 二等学业奖学金
  \item 2018 一等学业奖学金
  \item 2019 二等学业奖学金
  \item 2017-2018 三好研究生
\end{itemize}

\datedsubsection{\textbf{北見工業大学}, 北海道, 日本}{2020年4月 -- 至今}
\begin{itemize}
  \item N2Women Young Researcher Fellowship at MONAMI (2020)
  \item 2022 北见市留学生修学支援金
  \item 2023 北见市留学生修学支援金
\end{itemize}

%% Reference
%\newpage
%\bibliographystyle{IEEETran}
%\bibliography{mycite}
\end{document}
